\documentclass{report}

% Include necessary LaTeX packages for math
\usepackage{amsmath} % for advanced math symbols and environments
\usepackage{amsfonts} % for additional math fonts
\usepackage{amssymb} % for additional math symbols
\usepackage{mathtools} % for additional math tools and symbols
\usepackage{bm} % for bold math symbols
\usepackage{mathrsfs} % for additional math fonts

% Set up document formatting (optional)
\usepackage{geometry} % for adjusting page margins and size
\usepackage{setspace} % for adjusting line spacing
%\usepackage{enumitem} % for customizing lists



\newtheorem{definition}{\textcolor{blue}{Definition}}
\newtheorem{theorem}{THEOREM}[chapter]
\newtheorem{example}{\textit{Example}}
\newtheorem{fact}{Fact}[chapter]
\newtheorem{proposition}{Proposition}
\newtheorem{lemma}{Lemma}[section]
\newtheorem{conjecture}{Conjecture}
\newtheorem{axiom}{AXIOM}[chapter]

\newtheorem{problem}{Problem}[chapter]
\newtheorem{solution}{Solution}[chapter]

\begin{document}


Topics to be covered:

\begin{enumerate}

\item  Introduction to Abstract Algebra and Set Theory (1 week)
    
       Basic set operations, relations, and functions

       Cartesian products and equivalence relations

\item   Groups (3 weeks)
    
       Definition and examples

       Subgroups, cyclic groups, and permutation groups

       Cosets, normal subgroups, and factor groups

\item   Group Homomorphisms and Isomorphism Theorems (1.5 weeks)
    
       Group homomorphisms, kernels, and images

       First, Second, and Third Isomorphism Theorems

\item   Rings (2.5 weeks)
    
       Definition and examples

       Subrings, ideals, and quotient rings

       Ring homomorphisms and isomorphisms


\item  Fields and Field Extensions (2 weeks)
    
       Definition and examples

       Algebraic and transcendental extensions

       Finite fields and the Frobenius automorphism

\item  Introduction to Galois Theory (2 weeks)
    
       Fundamental theorem of Galois theory

       Solvability by radicals and the unsolvability of the quintic

       Applications to classical geometric constructions

\item  Module Theory (1 week)
    
       Definition and examples

       Submodules, quotient modules, and direct sums

       Module homomorphisms and isomorphisms

\end{enumerate}

\documentclass{report}

% Include necessary LaTeX packages for math
\usepackage{amsmath} % for advanced math symbols and environments
\usepackage{amsfonts} % for additional math fonts
\usepackage{amssymb} % for additional math symbols
\usepackage{mathtools} % for additional math tools and symbols
\usepackage{bm} % for bold math symbols
\usepackage{mathrsfs} % for additional math fonts

% Set up document formatting (optional)
\usepackage{geometry} % for adjusting page margins and size
\usepackage{setspace} % for adjusting line spacing
%\usepackage{enumitem} % for customizing lists



\newtheorem{definition}{\textcolor{blue}{Definition}}
\newtheorem{theorem}{THEOREM}[chapter]
\newtheorem{example}{\textit{Example}}
\newtheorem{fact}{Fact}[chapter]
\newtheorem{proposition}{Proposition}
\newtheorem{lemma}{Lemma}[section]
\newtheorem{conjecture}{Conjecture}
\newtheorem{axiom}{AXIOM}[chapter]

\newtheorem{problem}{Problem}[chapter]
\newtheorem{solution}{Solution}[chapter]


\begin{document}

\chapter{Introduction to Abstract Algebra and Set Theory}

\section{Basic Set Operations, Relations, and Functions}

In this section, we will cover the basics of set theory, including set operations, relations, and functions. These concepts are fundamental to understanding abstract algebra.

\subsection{Set Operations}

A \textit{set} is a collection of distinct objects, called \textit{elements} or \textit{members}. We usually denote sets using capital letters, like $A$, $B$, and $C$. If an element $x$ belongs to a set $A$, we write $x \in A$. If $x$ does not belong to $A$, we write $x \notin A$.

There are several basic operations on sets that we will use throughout this course:

\begin{itemize} 

\item \textbf{Union:} The union of two sets $A$ and $B$, denoted $A \cup B$, is the set containing all elements that are in either $A$ or $B$ or both. In other words, $A \cup B = \{x:x \in A \text{ or } x \in B\}$. 

    \item \textbf{Intersection:} The intersection of two sets $A$ and $B$, denoted $A \cap B$, is the set containing all elements that are in both $A$ and $B$. In other words, $A \cap B = \{x:x \in A \text{ and } x \in B\}$.

    \item \textbf{Difference:} The difference of two sets $A$ and $B$, denoted $A \setminus B$, is the set containing all elements that are in $A$ but not in $B$. In other words, $A \setminus B = {x:x \in A \text{ and } x \notin B}$. 

    \item \textbf{Complement:} The complement of a set $A$, denoted $A^c$ or $\overline{A}$, is the set containing all elements that are not in $A$. The complement operation requires a \textit{universal set} $U$ as a reference, which contains all the elements under consideration. In other words, $A^c ={x:x \in U \text{ and } x \notin A}$.

    \item \textbf{Cartesian Product:} The Cartesian product of two sets $A$ and $B$, denoted $A \times B$, is the set of all ordered pairs $(a, b)$, where $a \in A$ and $b \in B$. In other words, $A \times B = \{(a,b):a \in A \text{ and } b \in B\}$. 

\end{itemize}

\subsection{Relations}

A \textit{relation} is a set of ordered pairs of elements. More specifically, a relation $R$ on two sets $A$ and $B$ is a subset of their Cartesian product $A \times B$. We say that $a \in A$ is related to $b \in B$ by the relation $R$ if $(a, b) \in R$. We write this as $aRb$.

Some common types of relations are:

\begin{itemize} 

\item \textbf{Equivalence Relation:} A relation $R$ on a set $A$ is an equivalence relation if it satisfies the following three properties: 

\begin{enumerate} 

    \item \textit{Reflexivity:} For all $a \in A$, $aRa$. In other words, every element is related to itself. 
    
    \item \textit{Symmetry:} For all $a, b \in A$, if $aRb$, then $bRa$. In other words, if $a$ is related to $b$, then $b$ is related to $a$. 
    
    \item \textit{Transitivity:} For all $a, b, c \in A$, if $aRb$ and $bRc$, then $aRc$. In other words, if $a$ is related to $b$ and $b$ is related to $c$, then $a$ is related to $c$. 
\end{enumerate}


\item \textbf{Partial Order:} A relation $R$ on a set $A$ is a partial order if it satisfies the following three properties: 
    \begin{enumerate} 
        
        \item \textit{Reflexivity:} For all $a \in A$, $aRa$. 
        
        \item \textit{Antisymmetry:} For all $a, b \in A$, if $aRb$ and $bRa$, then $a = b$. In other words, if $a$ is related to $b$ and $b$ is related to $a$, then $a$ and $b$ are equal. 
        
        \item \textit{Transitivity:} For all $a, b, c \in A$, if $aRb$ and $bRc$, then $aRc$. 
        \end{enumerate} 

    \end{itemize}

\subsection{Functions}

A \textit{function} is a special kind of relation between two sets $A$ and $B$. A function $f$ from a set $A$ to a set $B$, denoted $f : A \to B$, is a rule that assigns to each element $a \in A$ a unique element $b \in B$. We write $f(a) = b$ to indicate that the function $f$ maps the element $a$ to the element $b$. The set $A$ is called the \textit{domain} of the function, and the set $B$ is called the \textit{codomain}. The \textit{range} of the function is the set of all elements in $B$ that are the images of elements in $A$. In other words, the range of $f$ is the set ${f(a) : a \in A}$.

A function can be can be categorised based on its properties:

\begin{itemize}
    \item A function $f : A \to B$ is called an \textit{injection} or \textit{one-to-one} if every element in $A$ is mapped to a distinct element in $B$. In other words, if $f(a_1) = f(a_2)$, then $a_1 = a_2$.

    \item A function $f : A \to B$ is called a \textit{surjection} (or \textit{onto}) if every element in $B$ is the image of at least one element in $A$. In other words, for every $b \in B$, there exists an $a \in A$ such that $f(a) = b$.

    \item A function $f : A \to B$ is called a \textit{bijection} or has a \textit{one-to-one correspondence} if it is both an injection and a surjection. In other words, $f$ is a bijection if every element in $A$ is mapped to a distinct element in $B$, and every element in $B$ is the image of at least one element in $A$. Bijections are important because they establish a strong correspondence between the elements of $A$ and $B$, indicating that the two sets have the same size or cardinality.
% check this definition...

\end{itemize}

\subsection{Composition of Functions}

Given two functions $f : A \to B$ and $g : B \to C$, we can define a new function called the \textit{composition} of $f$ and $g$, denoted as $g \circ f : A \to C$. The composition of $f$ and $g$ is defined as follows:

\[
(g \circ f)(a) = g(f(a)) \text{ for all } a \in A
\]

In other words, to compute the value of the composition at $a \in A$, we first apply the function $f$ to $a$, then apply the function $g$ to the result.

\subsection{Inverse Functions}

A function $f : A \to B$ has an \textit{inverse} if there exists a function $g : B \to A$ such that both $g \circ f = \text{id}\_A$ and $f \circ g = \text{id}\_B$, where $\text{id}\_A$ and $\text{id}\_B$ are the identity functions on $A$ and $B$, respectively. The identity function on a set $X$, denoted $\text{id}\_X$, is defined as $\text{id}\_X(x) = x$ for all $x \in X$. If such a function $g$ exists, we denote it as $f^{-1}$ and call it the inverse of $f$.
% fix this definition.

It is important to note that not all functions have inverses. For a function to have an inverse, it must be a bijection. In other words, the function must be both an injection and a surjection. If a function is a bijection, its inverse is unique.

This concludes the first section of Chapter 1 on basic set operations, relations, and functions. In the next section, we will explore Cartesian products and equivalence relations, which are essential tools for understanding more advanced concepts in abstract algebra.


\section{Cartesian Products and Equivalence Relations}

\subsection{Cartesian Products}

The Cartesian product of two sets $A$ and $B$, denoted as $A \times B$, is the set of all ordered pairs $(a, b)$, where $a \in A$ and $b \in B$. More formally,
\[
A \times B = \{(a, b) \mid a \in A, b \in B\}.
\]

\begin{example}

Let $A = \{1, 2\}$ and $B = \{a, b\}$. The Cartesian product $A \times B$ is

\[
A \times B = \{(1, a), (1, b), (2, a), (2, b)\}.
\]

\end{example}

The Cartesian product can be generalized to more than two sets. Given $n$ sets, the Cartesian product is the set of all ordered $n$-tuples.

\subsection{Relations}
A relation $R$ on a set $A$ is a subset of $A \times A$. If $(a, b) \in R$, we write $aRb$ or $a \sim b$. A relation can have the following properties:

\begin{itemize}
\item \textbf{Reflexive}: A relation $R$ is reflexive if for all $a \in A$, $aRa$. In other words, $(a, a) \in R$.
\item \textbf{Symmetric}: A relation $R$ is symmetric if for all $a, b \in A$, $aRb$ implies $bRa$. In other words, if $(a, b) \in R$, then $(b, a) \in R$.
\item \textbf{Transitive}: A relation $R$ is transitive if for all $a, b, c \in A$, $aRb$ and $bRc$ imply $aRc$. In other words, if $(a, b) \in R$ and $(b, c) \in R$, then $(a, c) \in R$.
\end{itemize}

\subsection{Equivalence Relations}
An equivalence relation is a relation that is reflexive, symmetric, and transitive. If a relation $\sim$ is an equivalence relation on a set $A$, then we can partition $A$ into disjoint subsets called equivalence classes.

%\begin{definition}
Definition:

Given an equivalence relation $\sim$ on a set $A$, the equivalence class of an element $a \in A$ is the set of all elements in $A$ that are related to $a$. It is denoted by $[a]$:

\[
[a] = \{x \in A \mid x \sim a\}.
\]

%\end{definition}

\begin{theorem}
If $\sim$ is an equivalence relation on a set $A$, then the equivalence classes partition $A$ into disjoint subsets.
\end{theorem}

%\begin{proof}

Proof:

Let $A$ be a set with an equivalence relation $\sim$, and let $a, b \in A$. There are two cases to consider for the intersection of the equivalence classes $[a]$ and $[b]$:


\begin{enumerate}

\item If $a \sim b$, then for any $x \in [a]$, we have $x \sim a \sim b$. Since $\sim$ is symmetric, $b \sim a$, and by transitivity, $x \sim b$. Therefore, $x \in [b]$, and $[a] \subseteq [b]$. Similarly, we can show that $[b] \subseteq [a]$, and thus, $[a] = [b]$.

\item If $a \not\sim b$, then $[a] \cap [b] = \emptyset$. Suppose there exists an element $x \in [a] \cap [b]$. Then $x \sim a$ and $x \sim b$. By symmetry, $b \sim x$, and by transitivity, $a \sim b$. This contradicts our assumption that $a \not \sim b$. Therefore, the intersection must be empty.

\end{enumerate}

In both cases, the equivalence classes are either equal or disjoint, which implies that the equivalence classes partition $A$.
%\end{proof}

\begin{example}
Let $A = \mathbb{Z}$, and define an equivalence relation $\sim$ on $\mathbb{Z}$ such that $a \sim b$ if and only if $a \equiv b \pmod{3}$. The equivalence classes are:

\begin{align*}
[0] &= \{\dots, -6, -3, 0, 3, 6, \dots\}, \\
[1] &= \{\dots, -5, -2, 1, 4, 7, \dots\}, \\
[2] &= \{\dots, -4, -1, 2, 5, 8, \dots\}.
\end{align*}

These equivalence classes partition $\mathbb{Z}$ into disjoint subsets.
\end{example}



\end{document}


\chapter{Groups}
\section{Definitions and examples}

\subsection{Definition of a Group}
A \textbf{group} is a set $G$, equipped with a binary operation $*$, which satisfies the following four axioms:

\begin{enumerate}
  \item \textbf{Closure:} For all $a, b \in G$, the result of the operation $a * b$ is also in $G$.
  \item \textbf{Associativity:} For all $a, b, c \in G$, the equation $(a * b) * c = a * (b * c)$ holds.
  \item \textbf{Identity:} There exists an element $e \in G$ such that for every $a \in G$, the equations $e * a = a$ and $a * e = a$ hold.
  \item \textbf{Inverses:} For each element $a \in G$, there exists an element $a^{-1} \in G$ such that $a * a^{-1} = e$ and $a^{-1} * a = e$.
\end{enumerate}

\subsection{Examples of Groups}
Here, we present some basic examples of groups:

\begin{example}
  The set of integers $\mathbb{Z}$ with the operation of addition is a group. The identity element is 0, and the inverse of any integer $a$ is $-a$.
\end{example}

\begin{example}
  The set of non-zero rational numbers $\mathbb{Q}^*$ with the operation of multiplication is a group. The identity element is 1, and the inverse of any rational number $a$ is $\frac{1}{a}$.
\end{example}

\begin{example}
  The symmetric group $S_n$ consists of all possible permutations of $n$ distinct objects. The binary operation is the composition of permutations. The identity element is the identity permutation, and the inverse of a permutation is its inverse permutation.
\end{example}

\begin{example}
  The set of $n \times n$ invertible matrices with real entries, denoted by $GL_n(\mathbb{R})$, forms a group under matrix multiplication. The identity element is the $n \times n$ identity matrix, and the inverse of a matrix is its inverse matrix.
\end{example}

\begin{example}
  The cyclic group of order $n$, denoted by $C_n$, consists of the elements $\{e, a, a^2, \dots, a^{n-1}\}$, where $a^n = e$. The binary operation is the multiplication modulo $n$. The identity element is $e$, and the inverse of an element $a^k$ is $a^{n-k}$.
\end{example}

These are just a few examples of groups; many more can be found throughout mathematics.


\section{Subgroups, cyclic groups, and permutation groups}

\subsection{Subgroups}
%\begin{definition}
Definition:

  A \textbf{subgroup} of a group $(G, *)$ is a subset $H \subseteq G$ that is itself a group under the same operation $*$. In other words, $H$ is a subgroup of $G$ if it satisfies the following conditions:

  \begin{enumerate}
    \item The identity element $e \in H$.
    \item If $a, b \in H$, then $a * b \in H$ (closure under the group operation).
    \item If $a \in H$, then $a^{-1} \in H$ (closure under inverses).
  \end{enumerate}
%\end{definition}

\begin{example}
  In the group of integers $\mathbb{Z}$ under addition, the set of even integers is a subgroup, denoted by $2\mathbb{Z}$.
\end{example}

\subsection{Cyclic groups}
%\begin{definition}
Definition:

  A group $G$ is called a \textbf{cyclic group} if there exists an element $a \in G$ such that every element in $G$ can be expressed as a power of $a$. In this case, $a$ is called a \textbf{generator} of the group, and we write $G = \langle a \rangle$.

%\end{definition}

\begin{example}
  The group of integers $\mathbb{Z}$ under addition is a cyclic group, generated by either $1$ or $-1$.
\end{example}

\begin{example}
  The group of units modulo $n$, denoted by $(\mathbb{Z}/n\mathbb{Z})^*$, is a cyclic group if and only if $n$ is either $1$, $2$, $4$, a power of an odd prime, or twice a power of an odd prime. In these cases, the group has a generator called a \textbf{primitive root modulo $n$}.
\end{example}

\subsection{Permutation groups}
%\begin{definition}
Definition:

  A \textbf{permutation} of a set $X$ is a bijective function from $X$ to itself. The set of all permutations of a finite set $X = \{1, 2, \dots, n\}$ forms a group called the \textbf{symmetric group} on $n$ elements, denoted by $S_n$. The group operation is the composition of functions.

%\end{definition}

\begin{example}
  The symmetric group $S_3$ consists of the following six permutations:

  \begin{align*}
    \text{id} &= \begin{pmatrix} 1 & 2 & 3 \\ 1 & 2 & 3 \end{pmatrix}, \\
    \sigma_1 &= \begin{pmatrix} 1 & 2 & 3 \\ 2 & 1 & 3 \end{pmatrix}, \\
    \sigma_2 &= \begin{pmatrix} 1 & 2 & 3 \\ 1 & 3 & 2 \end{pmatrix}, \\
    \sigma_3 &= \begin{pmatrix} 1 & 2 & 3 \\ 3 & 2 & 1 \end{pmatrix}, \\
    \tau_1 &= \begin{pmatrix} 1 & 2 & 3 \\ 2 & 3 & 1 \end{pmatrix}, \\
    \tau_2 &= \begin{pmatrix} 1 & 2 & 3 \\ 3 & 1 & 2 \end{pmatrix}.
  \end{align*}
\end{example}

\subsection{Alternating groups}

%\begin{definition}
Definition:

  An \textbf{alternating group} is the group of even permutations of a finite set $X = \{1, 2, \dots, n\}$. It is a subgroup of the symmetric group $S_n$ and is denoted by $A_n$.
%\end{definition}

\begin{example}
  The alternating group $A_3$ consists of the following three even permutations:

  \begin{align*}
    \text{id} &= \begin{pmatrix} 1 & 2 & 3 \\ 1 & 2 & 3 \end{pmatrix}, \\
    \tau_1 &= \begin{pmatrix} 1 & 2 & 3 \\ 2 & 3 & 1 \end{pmatrix}, \\
    \tau_2 &= \begin{pmatrix} 1 & 2 & 3 \\ 3 & 1 & 2 \end{pmatrix}.
  \end{align*}
\end{example}

In this section, we have introduced the concepts of subgroups, cyclic groups, permutation groups, and alternating groups. These are fundamental building blocks for more advanced topics in group theory.


\section{Cosets, normal subgroups, and factor groups}

\subsection{Cosets}
\begin{definition}
  Let $G$ be a group, and let $H$ be a subgroup of $G$. A \textbf{left coset} of $H$ in $G$ is a set of the form $gH = \{gh : h \in H\}$, where $g \in G$. Similarly, a \textbf{right coset} of $H$ in $G$ is a set of the form $Hg = \{hg : h \in H\}$.
\end{definition}

\begin{example}
  In the group of integers $\mathbb{Z}$ under addition, consider the subgroup $H = 3\mathbb{Z}$ of multiples of 3. The left cosets of $H$ are $0 + H = H, 1 + H, 2 + H$, and the right cosets are $H + 0 = H, H + 1, H + 2$.
\end{example}

\subsection{Normal subgroups}
\begin{definition}
  A subgroup $N$ of a group $G$ is called a \textbf{normal subgroup} if the left and right cosets coincide for every element of $G$, i.e., $gN = Ng$ for all $g \in G$.
\end{definition}

\begin{example}
  In the group of integers $\mathbb{Z}$ under addition, all subgroups are normal subgroups because the operation is commutative.
\end{example}

\begin{example}
  In a group $G$, the trivial subgroup $\{e\}$ and the entire group $G$ are always normal subgroups.
\end{example}

\subsection{Factor groups}
\begin{definition}
  If $N$ is a normal subgroup of a group $G$, we can define a new group called the \textbf{factor group} or \textbf{quotient group} of $G$ by $N$, denoted by $G/N$. The elements of $G/N$ are the cosets of $N$ in $G$, and the group operation is defined as follows: $(g_1N)(g_2N) = (g_1g_2)N$ for any $g_1, g_2 \in G$.
\end{definition}

\begin{example}
  In the group of integers $\mathbb{Z}$ under addition, consider the subgroup $H = 3\mathbb{Z}$ of multiples of 3. The factor group $\mathbb{Z}/H$ consists of the cosets $0 + H, 1 + H, 2 + H$, and the operation is addition modulo 3.
\end{example}




\end{document}
