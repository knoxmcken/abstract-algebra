\section*{Answers and Explanations}

\begin{enumerate}
\item
Find the following:

\begin{enumerate}
    \item $A \times B = \{(1, a), (1, b), (2, a), (2, b), (3, a), (3, b)\}$
    \item $B \times A = \{(a, 1), (a, 2), (a, 3), (b, 1), (b, 2), (b, 3)\}$
    \item $A \times A = \{(1, 1), (1, 2), (1, 3), (2, 1), (2, 2), (2, 3), (3, 1), (3, 2), (3, 3)\}$
\end{enumerate}

\item
Determine whether the following relations on the set $\mathbb{Z}$ are reflexive, symmetric, and/or transitive:

\begin{enumerate}
    \item $R_1 = \{(a, b) \mid a \equiv b \pmod{5}\}$ is reflexive, symmetric, and transitive. This is because any integer $a$ is congruent to itself modulo 5, the congruence relation is symmetric by definition, and if $a \equiv b \pmod{5}$ and $b \equiv c \pmod{5}$, then $a \equiv c \pmod{5}$ by the transitive property of congruences.
    \item $R_2 = \{(a, b) \mid a \leq b\}$ is reflexive and transitive but not symmetric. This is because any integer $a$ is less than or equal to itself, and if $a \leq b$ and $b \leq c$, then $a \leq c$. However, if $a \leq b$, it is not necessarily true that $b \leq a$.
    \item $R_3 = \{(a, b) \mid a^2 = b^2\}$ is reflexive, symmetric, and transitive. This is because $a^2 = a^2$ for any integer $a$, if $a^2 = b^2$, then $b^2 = a^2$, and if $a^2 = b^2$ and $b^2 = c^2$, then $a^2 = c^2$.
\end{enumerate}

\item
The relation $R$ is reflexive because $(a, a) \in R$ for all $a \in A$. It is symmetric because whenever $(a, b) \in R$, we also have $(b, a) \in R$. However, it is not transitive because $(1, 2) \in R$ and $(2, 2) \in R$, but $(1, 2) \notin R$. Thus, $R$ is not an equivalence relation.

\item
First, we prove that $R$ is reflexive. For any integer $a$, $a - a = 0$, which is even. Thus, $aRa$ for all $a \in \mathbb{Z}$, and $R$ is reflexive.

Next, we prove that $R$ is symmetric. If $aRb$, then $a - b$ is even. Since the negative of an even integer is also even, $b - a$ is even, and $bRa$. Thus, $R$ is symmetric.


Finally, we prove that $R$ is transitive. If $aRb$ and $bRc$, then $a - b$ and $b - c$ are even integers. Let $a - b = 2m$ and $b - c = 2n$ for some integers $m$ and $n$. Then,

\[
a - c = (a - b) + (b - c) = 2m + 2n = 2(m + n).
\]

Since $a - c$ is also an even integer, we have $aRc$. Thus, $R$ is transitive.

Since $R$ is reflexive, symmetric, and transitive, it is an equivalence relation. The equivalence classes are:
\[
[0] = \{\dots, -6, -4, -2, 0, 2, 4, 6, \dots\} \quad \text{and} \quad [1] = \{\dots, -5, -3, -1, 1, 3, 5, 7, \dots\}.
\]

\item
To find the Cartesian product $A \times B \times A$, we can first compute $A \times B$ and then compute the product of the resulting set with $A$. We have:

\[
A \times B = \{(1, 4), (1, 5), (2, 4), (2, 5), (3, 4), (3, 5)\}.
\]

Now, we compute $A \times B \times A$:

\[
A \times B \times A = \{(1, 4, 1), (1, 4, 2), (1, 4, 3), (1, 5, 1), (1, 5, 2), (1, 5, 3), \\
(2, 4, 1), (2, 4, 2), (2, 4, 3), (2, 5, 1), (2, 5, 2), (2, 5, 3), \\
(3, 4, 1), (3, 4, 2), (3, 4, 3), (3, 5, 1), (3, 5, 2), (3, 5, 3)\}.
\]
\end{enumerate}

