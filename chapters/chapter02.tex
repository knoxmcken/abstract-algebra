\chapter{Groups}
\section{Definitions and examples}

\subsection{Definition of a Group}
A \textbf{group} is a set $G$, equipped with a binary operation $*$, which satisfies the following four axioms:

\begin{enumerate}
  \item \textbf{Closure:} For all $a, b \in G$, the result of the operation $a * b$ is also in $G$.
  \item \textbf{Associativity:} For all $a, b, c \in G$, the equation $(a * b) * c = a * (b * c)$ holds.
  \item \textbf{Identity:} There exists an element $e \in G$ such that for every $a \in G$, the equations $e * a = a$ and $a * e = a$ hold.
  \item \textbf{Inverses:} For each element $a \in G$, there exists an element $a^{-1} \in G$ such that $a * a^{-1} = e$ and $a^{-1} * a = e$.
\end{enumerate}

\subsection{Examples of Groups}
Here, we present some basic examples of groups:

\begin{example}
  The set of integers $\mathbb{Z}$ with the operation of addition is a group. The identity element is 0, and the inverse of any integer $a$ is $-a$.
\end{example}

\begin{example}
  The set of non-zero rational numbers $\mathbb{Q}^*$ with the operation of multiplication is a group. The identity element is 1, and the inverse of any rational number $a$ is $\frac{1}{a}$.
\end{example}

\begin{example}
  The symmetric group $S_n$ consists of all possible permutations of $n$ distinct objects. The binary operation is the composition of permutations. The identity element is the identity permutation, and the inverse of a permutation is its inverse permutation.
\end{example}

\begin{example}
  The set of $n \times n$ invertible matrices with real entries, denoted by $GL_n(\mathbb{R})$, forms a group under matrix multiplication. The identity element is the $n \times n$ identity matrix, and the inverse of a matrix is its inverse matrix.
\end{example}

\begin{example}
  The cyclic group of order $n$, denoted by $C_n$, consists of the elements $\{e, a, a^2, \dots, a^{n-1}\}$, where $a^n = e$. The binary operation is the multiplication modulo $n$. The identity element is $e$, and the inverse of an element $a^k$ is $a^{n-k}$.
\end{example}

These are just a few examples of groups; many more can be found throughout mathematics.


\section{Subgroups, cyclic groups, and permutation groups}

\subsection{Subgroups}
%\begin{definition}
Definition:

  A \textbf{subgroup} of a group $(G, *)$ is a subset $H \subseteq G$ that is itself a group under the same operation $*$. In other words, $H$ is a subgroup of $G$ if it satisfies the following conditions:

  \begin{enumerate}
    \item The identity element $e \in H$.
    \item If $a, b \in H$, then $a * b \in H$ (closure under the group operation).
    \item If $a \in H$, then $a^{-1} \in H$ (closure under inverses).
  \end{enumerate}
%\end{definition}

\begin{example}
  In the group of integers $\mathbb{Z}$ under addition, the set of even integers is a subgroup, denoted by $2\mathbb{Z}$.
\end{example}

\subsection{Cyclic groups}
%\begin{definition}
Definition:

  A group $G$ is called a \textbf{cyclic group} if there exists an element $a \in G$ such that every element in $G$ can be expressed as a power of $a$. In this case, $a$ is called a \textbf{generator} of the group, and we write $G = \langle a \rangle$.

%\end{definition}

\begin{example}
  The group of integers $\mathbb{Z}$ under addition is a cyclic group, generated by either $1$ or $-1$.
\end{example}

\begin{example}
  The group of units modulo $n$, denoted by $(\mathbb{Z}/n\mathbb{Z})^*$, is a cyclic group if and only if $n$ is either $1$, $2$, $4$, a power of an odd prime, or twice a power of an odd prime. In these cases, the group has a generator called a \textbf{primitive root modulo $n$}.
\end{example}

\subsection{Permutation groups}
%\begin{definition}
Definition:

  A \textbf{permutation} of a set $X$ is a bijective function from $X$ to itself. The set of all permutations of a finite set $X = \{1, 2, \dots, n\}$ forms a group called the \textbf{symmetric group} on $n$ elements, denoted by $S_n$. The group operation is the composition of functions.

%\end{definition}

\begin{example}
  The symmetric group $S_3$ consists of the following six permutations:

  \begin{align*}
    \text{id} &= \begin{pmatrix} 1 & 2 & 3 \\ 1 & 2 & 3 \end{pmatrix}, \\
    \sigma_1 &= \begin{pmatrix} 1 & 2 & 3 \\ 2 & 1 & 3 \end{pmatrix}, \\
    \sigma_2 &= \begin{pmatrix} 1 & 2 & 3 \\ 1 & 3 & 2 \end{pmatrix}, \\
    \sigma_3 &= \begin{pmatrix} 1 & 2 & 3 \\ 3 & 2 & 1 \end{pmatrix}, \\
    \tau_1 &= \begin{pmatrix} 1 & 2 & 3 \\ 2 & 3 & 1 \end{pmatrix}, \\
    \tau_2 &= \begin{pmatrix} 1 & 2 & 3 \\ 3 & 1 & 2 \end{pmatrix}.
  \end{align*}
\end{example}

\subsection{Alternating groups}

%\begin{definition}
Definition:

  An \textbf{alternating group} is the group of even permutations of a finite set $X = \{1, 2, \dots, n\}$. It is a subgroup of the symmetric group $S_n$ and is denoted by $A_n$.
%\end{definition}

\begin{example}
  The alternating group $A_3$ consists of the following three even permutations:

  \begin{align*}
    \text{id} &= \begin{pmatrix} 1 & 2 & 3 \\ 1 & 2 & 3 \end{pmatrix}, \\
    \tau_1 &= \begin{pmatrix} 1 & 2 & 3 \\ 2 & 3 & 1 \end{pmatrix}, \\
    \tau_2 &= \begin{pmatrix} 1 & 2 & 3 \\ 3 & 1 & 2 \end{pmatrix}.
  \end{align*}
\end{example}

In this section, we have introduced the concepts of subgroups, cyclic groups, permutation groups, and alternating groups. These are fundamental building blocks for more advanced topics in group theory.


\section{Cosets, normal subgroups, and factor groups}

\subsection{Cosets}
\begin{definition}
  Let $G$ be a group, and let $H$ be a subgroup of $G$. A \textbf{left coset} of $H$ in $G$ is a set of the form $gH = \{gh : h \in H\}$, where $g \in G$. Similarly, a \textbf{right coset} of $H$ in $G$ is a set of the form $Hg = \{hg : h \in H\}$.
\end{definition}

\begin{example}
  In the group of integers $\mathbb{Z}$ under addition, consider the subgroup $H = 3\mathbb{Z}$ of multiples of 3. The left cosets of $H$ are $0 + H = H, 1 + H, 2 + H$, and the right cosets are $H + 0 = H, H + 1, H + 2$.
\end{example}

\subsection{Normal subgroups}
\begin{definition}
  A subgroup $N$ of a group $G$ is called a \textbf{normal subgroup} if the left and right cosets coincide for every element of $G$, i.e., $gN = Ng$ for all $g \in G$.
\end{definition}

\begin{example}
  In the group of integers $\mathbb{Z}$ under addition, all subgroups are normal subgroups because the operation is commutative.
\end{example}

\begin{example}
  In a group $G$, the trivial subgroup $\{e\}$ and the entire group $G$ are always normal subgroups.
\end{example}

\subsection{Factor groups}
\begin{definition}
  If $N$ is a normal subgroup of a group $G$, we can define a new group called the \textbf{factor group} or \textbf{quotient group} of $G$ by $N$, denoted by $G/N$. The elements of $G/N$ are the cosets of $N$ in $G$, and the group operation is defined as follows: $(g_1N)(g_2N) = (g_1g_2)N$ for any $g_1, g_2 \in G$.
\end{definition}

\begin{example}
  In the group of integers $\mathbb{Z}$ under addition, consider the subgroup $H = 3\mathbb{Z}$ of multiples of 3. The factor group $\mathbb{Z}/H$ consists of the cosets $0 + H, 1 + H, 2 + H$, and the operation is addition modulo 3.
\end{example}


